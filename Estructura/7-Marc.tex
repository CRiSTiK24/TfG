\chapter{Marc de treball i conceptes previs}
\section{Marc de treball}

El projecte tal i com s'ha estipuilat al \autoref{cap:5} s'ha elaborat de manera autònoma amb suport del tutor.Al ser un projecte individual hi ha hagut llibertat a l'hora de escollir les eines amb les que treballar, en aquest cas com a editors de text s'han utilitzat \textit{Sublime Text} ja que és un editor força lleuger al que s'ha pogut anant guardant diferent documentació i \textit{Visual Studio Code} per a desenvolupar les parts més técniques del projecte, que té moltes funcionalitats útils per a projectes i és molt complert.

Com a navegador s'ha utilitzat \textit{Brave} que és un navegador basat en chromium amb una forta protecció contra rastrejadors i anuncis.


\section{Conceptes previs}

Durant la elaboració del treball s'han utilitzat coneixements de les diverses tecnologies i conceptes com a base.

\subsection{Xarxa Local i Internet}
\label{xarxa}

Durant el projecte es mencionaràn diversos conceptes com \textit{IPs}, \textit{Ports}, \textit{Xarxes Locals}, \textit{Routers}, \textit{NATs} i \textit{Internet}, els cuals és important coneixer el significat basic. Imaginem una xarxa molt simple amb tres ordinadors, per tal de comunicar-se i enviar-se informació els conectem tots amb tots i els assignem una identitat única en aquesta xarxa, a cada una dels extrems de les conexions. Aquesta identitat s'anomena \textit{IP}, i consisteix en una serie de 4 numeros separats amb un rang de 0 a 255 que estan separats per punts amb l'estandard \textit{IP}v4. Per enviar un paquet i rebre una resposta, són comuns els protocols \textit{TCP} i \textit{UDP}, on el primer permet una conexió més segura pero lenta, i el segon més ràpida, pero amb risc de perdua de informació.En el moment que es volen conectar més dispositius a la xarxa, hi ha el problema que les màquines no tenen capacitat per a tenir tantes interficies de xarxa (On habitualment es conectaria el cable \textit{Ethernet}) com disposius, a més que conectar tots amb tots fa que es necesiti molt de cable,i per això es van crear els seguents dos dispositius: \textit{Commutadors} i \textit{Routers}. El primer redirigeix tots els paquets que li arriben a tots els dispositius conectats a ell, mentre que el segon envia els paquets en base a regles internes on es decideix a on enviar-los segons la destinació del paquet. Aquests dispositius tenen molts ports Ethernet i estan especialitzats per a aquesta tasca de ser nexes d'una xarxa, a la vegada que es poden combinar flexiblement, assignant uns rangs de \textit{IPs} únics a cada xarxa, fins a eventualment enllaçar moltes xarxes amb rangs de \textit{IPs} úniques i crear el que coneixem com a \textit{Internet}.\cite{hunt_tcpip_2002}

Eventualment les \textit{IPs} es van anar esgotant, doncs al tenir una longitud de 32 bits, només poden tenir una ip única 4.294.967.296 de dispositius. Llavors mentre es desenvolupava un nou protocol IP anomenat \textit{IP}v6 amb un rang de $2^{128}$ direccions úniques, es va desenvolupar un métode per aliviar temporalment aquest problema anomenat \textit{NAT}. Aquest métode consisteix en assignar les \textit{IPs públiques}, es a dir les que són úniques a tot internet, només als routers de les xarxes més petites com les doméstiques. Llavors quan les màquines volguessin fer una consulta a internet, el router s'apunta el port i la \textit{IP local} d'origen juntament amb la \textit{IP pública} i port d'aquest que es vol accedir, per tal de que quan retorni un paquet desde aquesta ubicació es pugui redirigir aquesta a la màquina que sap que ha fet la petició. La \textit{IP local} és una adreça que s'assignen a lés màquines d'una xarxa que no està directament conectada a internet, sinò que l'access es fa a partir d'un router amb NAT. Les adreces d'una xarxa local que treballa amb \textit{IP}v4 en estan limitades a certs rangs, en concret de 10.0.0.0 fins a 10.255.255.255, 172.16.0.0 fins 172.31.255.255 i 192.168.0.0 fins 192.168.255.255. \cite{wu_transition_2013}

\subsection{Ports i tallafocs}

Un port és un número identificador assignat per a identificar un servei o lloc de connexió al qual es poden enviar o rebre dades. Aquest número està dins del rang de 0 fins a 65535, definit pel protocol \textit{TCP/IP} en la majoria de sistemes operatius. Hi ha alguns ports que solen ser estandarditzats, com ara el port 80 per a comunicacions \textit{HTTP} i el 443 per a \textit{HTTPS}. Aquests es coneixen com a \textit{ports ben coneguts} i van del 0 fins al 1023. Els altres ports es divideixen en \textit{ports registrats}, que van del 1024 fins al 49151 i estan registrats per l’organització \textit{IANA}, i en \textit{ports dinàmics}, que són utilitzats pels clients que volen obrir connexió amb un servidor i van del 49152 fins al 65535.

La comunicació entre màquines a Internet funciona mitjançant \textit{IPs} i ports, tal com s'ha explicat a la \autoref{xarxa}, i, per tant, consisteix en què els missatges arriben a la màquina per un port en què hi ha un programa anomenat servidor escoltant, el qual executa les instruccions del programa cada cop que arriba un missatge. També, a l'hora de comunicar-se amb un servidor remot, com per exemple fan els navegadors, es necessita un port que s'assigna dinàmicament dins del rang dels \textit{ports dinàmics}. Els sistemes operatius moderns apliquen moltes tècniques per optimitzar i millorar aquest procés i, per tant, és força més complex, però en general aquesta definició és suficient per a l'ambició del treball.\cite{benvenuti_understanding_2005}

\newpage
Per altra banda, un tallafocs, tal com el va definir \textit{Marcus Ranum}, el creador del primer tallafocs comercial: \say{Un tallafocs és la implementació de la teva política de seguretat a Internet.}
Aleshores, en què consisteixen els tallafocs en els sistemes operatius moderns? Són normes que no permeten l'accés a certs ports de la màquina, mentre que en permeten l'accés a altres. Hi ha alguns protocols que és perillós deixar oberts a Internet sense una configuració adequada, com per exemple el \textit{SSH}, un protocol que permet connectar-se a una màquina remota a través d'Internet, i per tant solen estar bloquejats per defecte. En cas de voler-ho canviar, caldrà fer canvis en aquestes normes predefinides pel sistema operatiu, ja sigui mitjançant un programa preinstal·lat que ho gestioni o bé amb un programa de tercers amb permisos per modificar els ports.

\subsection{Docker}
\label{Docker}

\textit{Docker} és una eina que permet contenir programes en màquines virtualitzades anomenades contenidors. Aquests contenidors es construeixen a partir d’una \textit{imatge}, una seqüència d'instruccions que indica quins passos s'han de seguir per a instanciar un \textit{contenidor} amb tot el programari necessari, incloent-hi el sistema operatiu reduït dins de la màquina virtual. \cite{merkel_docker_nodate}

Els coneixements de Docker, i d’eines habituals com \textit{Docker Compose}, que permeten gestionar fàcilment conjunts de contenidors, han estat necessaris perquè el back-end del joc senzill es troba disponible com una imatge Docker. Això vol dir que és un programa contingut en un contenidor, i que per interactuar amb ell s'ha de fer mitjançant els ports de l'ordinador.

El fet que el servidor estigui disponible com a imatge fa que sigui extremadament fàcil d’instal·lar a qualsevol sistema operatiu amb accés a Docker, ja que l’avantatge principal d’aquest sistema és la seva portabilitat, gràcies al fet que els programes estan continguts en els seus respectius contenidors.

\subsection{Processament multifil i Goroutines a k6}
Un procés és la encapsulació d'un programa en execució, de tal manera que té assignats uns recursos limitats pel sistema operatiu que pot utilitzar. Antigament, els processos s'executaven en un sol fil d'execució, que és una representació lògica d'una llista d'instruccions que ha de ser executada pel processador. A mesura que la tecnologia va avançar i es va trobar que no es podia incrementar la freqüència del nucli del processador per fer-lo més ràpid, a causa de l'augment de la calor dissipada, el focus va deixar de ser l'augment de la freqüència del processador. En canvi, es va centrar a fer el nucli més petit per permetre tenir-ne diversos en un mateix processador.

Aquest canvi en el món del \textit{hardware} va venir acompanyat de canvis en el \textit{software}, per tal de poder aprofitar els nuclis addicionals de manera efectiva, amb el que avui coneixem com a \textit{processament multifil}.\cite{its_processes_nodate}

El processament multifil consisteix a aprofitar al màxim la capacitat de reordenar les instruccions d'un programa sense alterar-ne el resultat i l'absència de dependències entre instruccions, cosa que permet calcular diverses parts simultàniament. Així doncs, el que es fa és dividir el programa en fragments i distribuir la càrrega de manera equitativa entre els nuclis disponibles. A més, molts processadors moderns tenen una separació lògica dins de cada nucli del processador per permetre el paral·lelisme, coneguda com a \textit{fils lògics}.\cite{nemirovsky_multithreading_2022}

Les \textit{goroutines} són una implementació de fils d'execució molt lleugers, utilitzada en l'eina de creació de tests de càrrega \textit{k6}. Les principals avantatges d'aquesta implementació són que permeten l'execució de diverses \textit{goroutines} en un sol fil d'execució estàndard del sistema operatiu, i són menys costoses computacionalment que els fils d'execució tradicionals. Aquestes característiques les fan molt atractives per a un programa com \textit{k6}, ja que pot simular molts usuaris, on cada usuari és una \textit{goroutine}, amb molta més facilitat.\cite{alchangian_introduction_2023}

\subsection{Gestors de paquets}
L'experiència usual d'un usuari d'ordinador no desenvolupador consisteix en tenir una màquina amb el sistema operatiu \textit{Windows}, i que per descarregar programes hagi de seguir el següent procediment:

\begin{enumerate} \item Obrir un navegador. \item Anar a un cercador si no s'ha obert per defecte. \item Escriure el nom del programa que es vol descarregar. \item Clicar a la primera pàgina que no sigui un anunci. \item Buscar a la pàgina el botó de descarregar. \item Escollir la versió del programa segons l'arquitectura del processador (32 o 64 bits). \item Obrir l'explorador d'arxius. \item Buscar el programa a la carpeta de Baixades. \item Clicar en el programa descarregat. \item Seguir les instruccions que demani el programa durant la instal·lació. \end{enumerate}

Aquest procediment permet a l'usuari descarregar programari de tot Internet de manera senzilla, però és un procés molt lent, que requereix una interfície gràfica i molta interacció per part de l'usuari. Altres sistemes operatius, majoritàriament derivats de \textit{Linux} com \textit{Android}, \textit{Ubuntu} i \textit{Arch Linux}, utilitzen un sistema alternatiu anomenat \textit{Sistema de Gestió de Paquets}.

Aquest sistema es basa en la centralització de tots els programes disponibles en un sol repositori de paquets, els quals són mantinguts per voluntaris o pels desenvolupadors de les aplicacions mateixes. La manera d'interactuar amb aquest repositori i instal·lar-ne el programari és mitjançant el \textit{gestor de paquets} que el sistema operatiu en particular utilitza oficialment, per tal de garantir la millor experiència. En el cas d'\textit{Arch Linux}, és \textit{Pacman}; en el cas d'\textit{Ubuntu}, és \textit{apt/apt-get}; i en el cas d'\textit{Android}, és \textit{Package Manager}, tot i que s'acostuma a utilitzar l'aplicació \textit{Play Store} per descarregar les aplicacions mitjançant interfície gràfica.

Tot i que \textit{Microsoft} ha intentat popularitzar el seu propi \textit{gestor de paquets} anomenat \textit{winget}, amb \textit{Microsoft Store} com a interfície gràfica, el \textit{gestor de paquets} que ha acabat sent més popular és \textit{Chocolatey}, gràcies al seu millor desenvolupament per a l'ús amb línia de comandes, a la forta comunitat que publica voluntàriament els paquets, etc. La facilitat per ser utilitzat amb línia de comandes és la gran virtut que fa que els \textit{gestors de paquets} siguin tan populars entre els desenvolupadors: permeten automatitzar molt fàcilment la instal·lació dels programes amb les seves dependències i, a més, asseguren que es puguin reproduir fàcilment els errors, ja que tots els usuaris comparteixen el mateix paquet.\cite{spinellis_package_2012}

\subsection{Systemd}
A Linux, un procés molt popular a la majoria de distribucions és \textit{Systemd}, que s'executa en iniciar la màquina i s'encarrega d'iniciar altres programes, muntar els sistemes de fitxers, etc. El procés d'arrencada d'una màquina amb Linux i \textit{Systemd} és el següent\cite{both_systemd_2024}:

\begin{enumerate} \item Engegada del maquinari: Es fa mitjançant \textit{BIOS} o \textit{UEFI}. \item Engegada del nucli de Linux i \textit{Systemd}: Es fa mitjançant \textit{Linux boot}. \item Inici de la resta del sistema operatiu: Això ho gestiona \textit{Systemd}. \end{enumerate}

L'ús pràctic més comú per a la majoria d'usuaris de \textit{Systemd} és afegir els seus propis programes a l'inici de la resta del sistema operatiu. Això es fa mitjançant la comanda \textit{systemctl}, i hi ha molts paquets que faciliten l'ús d'aquesta eina amb \textit{Systemd}. Alternativament, es pot escriure manualment un fitxer a \textit{/etc/systemd/system/nomDelPrograma.service}, oferint molta flexibilitat sobre les instruccions que es volen executar en iniciar la resta del sistema operatiu.\cite{morel_creating_2019}