\chapter{Estudi de viabilitat}

Per tal de portar a la pràctica aquesta anàlisi, és necessari disposar de tres màquines. La primera serà l'ordinador que farà les peticions com a client, simulant una càrrega per als servidors. La segona pot ser qualsevol ordinador o portàtil vell que es tingui a disposició, que farà de servidor físic. La tercera, per altra banda, serà una màquina llogada al núvol.Per a aquest estudi en concret, s'ha triat un ordinador principal com a màquina client, és a dir, una màquina utilitzada diàriament que és prou potent i nova per a ser capaç d'enviar i rebre grans quantitats de dades als servidors sense que ens haguem de preocupar que faci de coll d'ampolla. En concret, la màquina disposa de les següents característiques:

\begin{table}[ht]
\centering
\begin{tabular}{ |p{3cm}||p{3cm}|p{3cm}|p{3cm}|  }
 \hline
 \multicolumn{3}{|c|}{Màquina Client} \\
 \hline
 Components & Nom & Rendiment \\
 \hline
 CPU  & i5-8400 & 2.8GHz x 6 nuclis\\
 RAM & 2x8GB  & 2133MHz   \\
 Ethernet & Cat 5e & 100Mbps\\
 \hline
\end{tabular}
\caption{Característiques principals de la Màquina Client}
\label{Tab:MaquinaClient}
\end{table}

Per a la segona màquina, s'ha escollit un portàtil vell amb un valor actual al voltant dels 250€, \textit{l'Acer Aspire E5-521 62DW}. El motiu principal per escollir aquesta màquina és la seva similitud amb altres portàtils o ordinadors disponibles per a una gran part de la població, els quals no se'ls dona gaire ús a causa de la seva feblesa amb les noves versions de Windows. Així que és interessant descobrir com de útils podrien ser com a servidors físics a casa.

\begin{table}[ht]
\centering
\begin{tabular}{ |p{3cm}||p{3cm}|p{3cm}|p{3cm}|  }
 \hline
 \multicolumn{3}{|c|}{Màquina Servidor Física} \\
 \hline
 Components & Nom & Rendiment \\
 \hline
 CPU  & AMD A6-6310 &  2 GHz x 4 nuclis\\
 RAM & 2x4GB  &  1600MHz   \\ 
 Ethernet & AR9565 Wireless & 65 Mbps\\ 
 \hline
\end{tabular}
\caption{Característiques principals de la Màquina Servidor Física}
\label{Tab:MaquinaServidorFisic}
\end{table}

Per a la tercera màquina, que ha de ser al núvol, s'ha escollit la \textit{e2-micro} a GCP. El motiu per escollir GCP és perquè prèviament a aquest projecte ja es tenia experiència bàsica amb Azure i AWS. Llavors es volia agafar una altra plataforma per estar el menys familiaritzat possible i així poder documentar les diverses dificultats. Pel que fa a la màquina en específic, la e2-micro és una màquina suficientment competent per a operacions senzilles, amb un lloguer que costa al voltant de 7,30€ al mes. La intenció amb aquesta màquina és poder comparar una màquina barata en el món del núvol amb la màquina anteriorment mostrada, la qual també és molt assequible.

\begin{table}[ht]
\centering
\begin{tabular}{ |p{3cm}||p{3cm}|p{3cm}|p{3cm}|  }
 \hline
 \multicolumn{3}{|c|}{Màquina Servidor Núvol} \\
 \hline
 Components & Nom & Rendiment \\
 \hline
 CPU  & Intel Broadwell &  2.25GHz x 2 nuclis x 12,5\% del temps\\
 RAM & 1GB  &  2000MHz   \\ %TODO: no hi ha manera de sabrer-ho
 Ethernet & Ethernet & 1000 Mbps\\ 
 \hline
\end{tabular}
\caption{Característiques principals de la Màquina Servidor Núvol}
\label{Tab:MaquinaServidorNuvul}
\end{table}

Pot sorprendre que el rendiment de la CPU tingui un percentatge del temps. Això es deu al fet que aquesta màquina en particular té accés a dos nuclis, però només se li atorga $\frac{1}{8}$ part de segon de cada un dels nuclis. Per tant, és gairebé equivalent al 25\% d'una CPU d'un sol nucli que va a una freqüència de 2.25GHz, o a tot el temps d'un nucli amb una freqüència de 0.56GHz.
