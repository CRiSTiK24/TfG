\chapter{Conclusions}

En aquest treball s'ha desenvolupat i explicat com instal·lar un servidor amb un joc senzill tant a la Màquina al núvol com a la Màquina local, i com aconseguir que altres màquines d'Internet es puguin connectar a elles mitjançant permisos als tallafocs i la redirecció de ports. Posteriorment, s'ha elaborat un codi senzill per realitzar diversos tests de càrrega diferents, els quals s'han pogut analitzar tant des de la perspectiva de la Màquina-Client mitjançant l'eina \textit{k6}, com des de la perspectiva de les Màquines-Servidor mitjançant l'eina \textit{Netdata}.

L'anàlisi dels resultats ha posat de manifest diversos aspectes sobre com està configurada la Màquina al núvol en el moment de rebre una gran càrrega, on, en cap de les proves, ha procedit a enviar una resposta amb error. També s'ha pogut observar que la principal limitació d'aquesta ha estat la seva baixa capacitat de processament, fet que la fa inadequada per fer de servidor per a un joc massiu senzill, ja que provoca una latència elevada ($\geq 500ms$) en el 34,5\% de les peticions, com es pot veure al Test 3 amb fins a 2.000 usuaris simultanis simulats. D'altra banda, els resultats han afavorit força la Màquina local, que en totes les proves ha demostrat un rendiment superior, especialment en la rapidesa de resposta en les peticions. Tot i això, també s'han identificat colls d'ampolla evidents, especialment pel que fa a l'amplada de banda, on s'ha pogut veure que, tot i que la màquina té més capacitat per processar dades, la seva capacitat de pujada d'informació a internet no li ha permès assolir el seu màxim rendiment.

Per finalitzar, s'ha constatat que tant la configuració de la Màquina local com la de la Màquina al núvol tenen la seva complexitat. La Màquina local ha requerit un coneixement més específic dels dispositius de la xarxa on està ubicada, com el router, mentre que la configuració al núvol ha necessitat molta més exploració en panells plens d'informació i precaució per assegurar-se que els recursos consultats estaven actualitzats. Atès que la complexitat d'una i altra Màquina ha estat equiparable, el factor determinant per triar amb quina de les dues procedir hauria de dependre de la quantitat de recursos que necessiti el servidor que es vol allotjar. Si es necessita un servidor on la càrrega de treball al processador és mínima, però es requereix molta amplada de banda, una màquina al núvol com la utilitzada en aquest treball podria ser l'opció ideal. En canvi, si es busca un servidor on no s'hagin d'enviar moltes dades, però es requereixi molt processament o memòria, llavors una màquina local com la descrita en aquest treball seria més adequada.