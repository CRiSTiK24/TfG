\chapter{Treball futur}

Expansions d'aquest estudi i treball es poden centrar en els següents apartats: \begin{itemize} 
\item Realitzar més proves amb aquestes dues màquines. 
\item Repetir els experiments, però amb la màquina local connectada per cable \textit{Ethernet} en comptes de \textit{Wi-Fi}. 
\item Canviar la màquina al núvol per una altra amb més capacitat de processament i memòria. 
\item Experimentar amb altres programes que facin de servidor d'un joc senzill. 
\item Utilitzar altres eines per a fer tests de càrrega i comparar el seu rendiment amb \textit{k6}. 
\item Adaptar el joc senzill actual perquè pugui ser utilitzat com a \textit{AWS Lambda}, \textit{Google Cloud Functions} o \textit{Azure Functions} al núvol, una arquitectura alternativa a una màquina virtual on, en comptes de cobrar pel temps que la màquina està encesa, es cobra per la quantitat de peticions realitzades. 
\item Fer que més d'una Màquina actuï com a client simultàniament per aproximar-se millor a una càrrega distribuïda real. 
\item Traslladar la Màquina-Servidor local a una xarxa d'una altra companyia de telecomunicacions ubicada a la mateixa ciutat, per observar si hi ha variabilitat en la connexió entre la Màquina-Client i la Màquina-Servidor local. 
\item Experimentar amb diferents patrons de càrrega en lloc d'utilitzar increments lineals.
\end{itemize}