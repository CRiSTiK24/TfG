\chapter{Requisits del sistema}

Com que aquest projecte no és un desenvolupament convencional de programari, els requisits es definiran per separat per a les següents parts: Màquina-Servidor, Màquina-Client i Test de Càrrega.

\section{Requisits d'una Màquina-Servidor}

Els requisits funcionals d'una Màquina-Servidor inclouen la llista de requisits imprescindibles per garantir l'accés al servidor des del Test de Càrrega de la Màquina-Client. Aquests són els següents:

\begin{itemize}
    \item La màquina ha de tenir un maquinari mínim per poder utilitzar el sistema operatiu modern que tingui instal·lat.
    \item La Màquina ha de tenir accés a Internet. 
    \item La Màquina no pot estar sota més d'un NAT\cite{wing_network_2010}, com ara un NAT massiu.\cite{richter_multi-perspective_2016} \item La Màquina ha de tenir un tallafoc instal·lat.
    \item Els ports necessaris per a la comunicació del servidor han d'estar oberts al tallafoc.\cite{liang_evolution_2022}
    \item Els ports han d'estar redirigits al router per permetre l'accés al port de la màquina des d'Internet. 
    \item La Màquina ha de tenir instal·lat i funcionant Docker Engine.  \cite{bashari_rad_introduction_2017}
\end{itemize}

Els requisits no funcionals d'una Màquina-Servidor inclouen aquells que milloren la qualitat de la prova, però no són estrictament necessaris:

\begin{itemize} 
    \item La Màquina ha de tenir components equilibrats per evitar qualsevol coll d'ampolla que pugui limitar el rendiment. \cite{singh_bottleneck_2012}
    \item La Màquina ha de tenir una connexió a Internet suficientment bona per gestionar el trànsit de dades. 
\end{itemize}

En aquest projecte, els requisits no funcionals es veuen limitats per les màquines utilitzades per a l'estudi. Per tant, aquests requisits no funcionals il·lustren les qualitats ideals d'una Màquina-Servidor, com ara una potència equilibrada i una bona connexió de xarxa.

\section{Requisits d'una Màquina-Client}

Els requisits funcionals d'una Màquina-Client inclouen els requisits imprescindibles per executar el Test de Càrrega i enviar paquets a les Màquines-Servidors. Aquests són els següents:

\begin{itemize}
    \item La Màquina ha de tenir accés a Internet. 
    \item La Màquina ha de tenir instal·lada l'eina \textit{k6}.
\end{itemize}

Els requisits no funcionals d'una Màquina-Client inclouen aquells que milloren la qualitat de la prova, però no són estrictament necessaris:

\begin{itemize} 
    \item La Màquina ha de tenir components equilibrats per evitar qualsevol coll d'ampolla que pugui limitar el rendiment. 
    \item La Màquina ha de tenir una CPU suficient per crear els \textit{Usuaris Virtuals} necessaris per a la paral·lelització de la càrrega. 
    \item La Màquina ha de tenir una connexió a Internet suficientment bona per gestionar les peticions enviades i rebudes. 
\end{itemize}

En aquest cas, és essencial que es compleixin els requisits no funcionals, ja que qualsevol limitació de la Màquina-Client podria afectar la qualitat de les dades obtingudes del rendiment dels servidors.

\section{Requisits d'un Test de Càrrega}

Els requisits funcionals d'un Test de Càrrega inclouen els requisits imprescindibles per permetre una anàlisi adequada d'un servidor HTTPS a Internet:

\begin{itemize} 
    \item El Test ha de contenir instruccions per enviar paquets al port HTTPS d'una màquina a Internet.
    \item El Test ha de retornar informació sobre les peticions fetes al servidor.
\end{itemize}

Els requisits no funcionals d'un Test de Càrrega inclouen aquells que milloren la qualitat de la prova, però no són estrictament necessaris:

\begin{itemize} 
    \item El Test ha de realitzar peticions que generin una càrrega de treball considerable. 
    \item El test ha de contenir tant contingut que no modifiqui l'estat del servidor, com contingut que sí que ho faci.
\end{itemize}